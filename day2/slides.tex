\documentclass[pdf]{beamer}
\mode<presentation>{}
\usetheme{Rochester}
\usecolortheme{whale}
\beamertemplatenavigationsymbolsempty

\usepackage{fontspec}
\setmainfont[ItalicFont={brilli.ttf}, BoldFont={brillb.ttf}]{brill.ttf}
\usefonttheme{serif}
\usepackage{microtype}
\usepackage{multicol}

\newcommand{\Subitem}[1]{{\setlength\itemindent{12pt} \item[-] #1}}
\newcommand{\Subsubitem}[1]{{\setlength\itemindent{24pt} \item[○] #1}}

\title{Meeting 2: Yóga, múnis, and orthodoxy}
\subtitle{The Vedic predecessors of a classical tradition}
\author{Nikhil Surya Dwibhashyam, Rohan Pandey}
\date{27 February 2022}

\begin{document}

\frame{\titlepage}

\begin{frame} \frametitle{Meeting agenda}
\begin{itemize}
	\item (Re)introduction to discussion group
	\item Today's meeting topic
	\item Free discussion
\end{itemize}
\end{frame}

\begin{frame} \frametitle{Why a Vedá discussion group?}
\begin{itemize}
	\item What is Vedic literature?
	\Subitem Sáṁhitā-s
	\Subitem Brā́hmaṇa-s
	\Subitem Others?
	\pause \item Vedá-s vs.~Vedā́nta \& Upaniṣád-s
	\Subitem Later monism (ádvāita) vs.~earlier dualism (dvāitá)
	\item Modern (religious, scholarly) fixation upon later literature
	\pause \item What is there to discuss?
	\Subitem History \& linguistics
	\Subitem Metaphysics of religion
	\Subitem Exegesis (e.g.~countering western narratives)
	\Subitem Moral principles
	\Subitem Nuances of Ṡrāutá ritual
\end{itemize}
\end{frame}

\begin{frame} \frametitle{Our plans}
\begin{itemize}
	\item Weekly meetings
	\item Different topic/aspect/angle discussed in depth every meeting
	\item Format:
	\Subitem Præsentations
	\Subitem Free-form discussions
	\item Eventually something more?
\end{itemize}
\end{frame}

\begin{frame} \frametitle{Let's introduce ourselves!}
\begin{itemize}
	\item Name
	\item School (if student)
	\item How you found this group
	\item Background/interest in Hinduism \& Vedic literature
\end{itemize}
\end{frame}

\begin{frame}[label=questions] \frametitle{Some questions to consider}
\begin{itemize}
	\item What \emph{is} Yóga exactly?
	\item What forerunners (if any) of what we know today as Yóga can be found in the Vedic religion?
	\Subitem To what extent were these forerunners already mainstream?
	\item To what extent is Yóga a uniquely post-Vedic or even Ṡramaṇá belief-system?
	\Subitem Can Yóga be synthesized with orthodox Vedic tradition?
	\item What biases in western Indology can be noticed from western analysis of Yóga?
	\item We'll return to these at the end!
\end{itemize}
\end{frame}

\begin{frame} \frametitle{What is Yóga?}
\begin{itemize}
	\item In the Vedá-s:
\end{itemize}
 
\begin{block} {yogá m.~(MW)}
\begin{itemize}
	\item the act of yoking, joining, attaching, harnessing, putting to (of horses) (R̥V)
	\item a yoke, team, vehicle, conveyance (ṠB)
	\item employment, use, application, performance (R̥V)
	\item undertaking, business, work (R̥V, AV, TS)
\end{itemize}
\end{block}
\end{frame}

\begin{frame} \frametitle{What is Yóga?}
\begin{itemize}
	\item In the Vedá-s:
\end{itemize}

\begin{block} {yogá m.~(Graßmann)}
\begin{itemize}
	\item das \emph{Anschirren} des Zugthieres oder Wagens [G.]
	\item \emph{Anschirrung}, \emph{Fahrt}
	\item bildlich: das \emph{Anschirren} d.~h.~\emph{zurüsten}, \emph{in Thätigkeit setzen} mit G.
	\item \emph{Unternehmung}, \emph{Werk}
	\item mit \textit{kṣéma}: \emph{Arbeit} und Ruhe
\end{itemize}
\end{block}
\end{frame}

\begin{frame} \frametitle{What is Yóga?}
\begin{itemize}
	\item Etymology: nominal derivation
	\Subitem Accent: nōmen āctiōnis vs.~agentis
\end{itemize}

\begin{block} {$\surd$yuj (MW)}
\begin{itemize}
	\item to yoke or join or fasten or harness (horses or a chariot) (R̥V \&c)
	\item to make ready, prepare, arrange, fit out, set to work, use, employ, apply (R̥V \&c)
	\item to fix in, insert, inject (ṠB)
	\item to turn or direct or fix or concentrate (the mind, thoughts \&c) upon (L.) (TS \&c)
	\item to join, unite, connect, add, bring together (R̥V \&c)
	\item to join one's self to (A.) (R̥V)
\end{itemize}
\end{block}
\end{frame}

\begin{frame} \frametitle{What is Yóga?}
\begin{itemize}
	\item Yóga as broadly understood in classical sense:
	\Subitem Discipline of the mind (cf.~TS radical)
	\Subitem Origins in at least Kaṭʰo-'paniṣád (ca.~Buddʰá)
	\Subitem 1 of 6 Hindu dárṡana-s
	\Subitem Expounded in Yoga-sū́tra-s of Patañjalí (ca. CE)
	\Subitem We'll focus on this today.
	\pause \item Yóga as understood in western sense:
	\Subitem Purely haṭʰa-yóga: from $\surd$\textit{haṭʰ} ``force''
	\Subsubitem Even more specifically: ā́sana-s
	\Subitem Much more modern! Perhaps even Western construct?
	\pause \item Point is: all decidedly post-Vedic!~or so it seems\ldots
\end{itemize}
\end{frame}

\begin{frame} \frametitle{Origins of contemplative tradition}
\begin{itemize}
	\item Famous creation hymn from R̥V:
\end{itemize}

\begin{center}
\begin{multicols}{2}
\scriptsize{
	N' ā́sad āsīn • n' ó sád āsīt tadā́nīm.

	Nā́ 'sīd rájo • n' ó víomā paró yat.

	Kím ā́ 'varīvaḥ? • Kúha? Kásya ṡárman?

	Ámbʰaḥ kím āsīd • gáhanaṁ gabʰīrám?

	\vspace{\baselineskip}

	Iyáṁ vísr̥ṣṭir • yáta ā babʰū́va,

	yádi vā dadʰé • yádi vā ná -- --,

	yáv asy' ā́dʰy-akṣaḥ • paramé víoman,

	sá aṅgá veda • yádi vā ná véda.

	\columnbreak

	\textit{The non-being was not, nor was the being then.}

	\textit{The sky was not, nor the heavens beyond it.}

	\textit{What moved it? Whither? In whose shelter?}

	\textit{Were there the waters impenetrable and deep?}

	\vspace{\baselineskip}

	\textit{He who was there from this creation,}

	\textit{whether he formed it or whether not,}

	\textit{the observer of this (world) in the highest heavens,}

	\textit{he indeed knows, or he knows not.}
}
\end{multicols}
 ---R̥V 10.129.1,7
\end{center}

\begin{itemize}
	\item (NB metrical lacuna.)
\end{itemize}
\end{frame}

\begin{frame} \frametitle{The Vedic múni-s: R̥V 10.136}
\begin{itemize}
	\item ``Yoga and the Ṛg Veda: An Interpretation of the Keśin Hymn.''
	\Subitem Werner, K. \emph{Religious Studies}, \textbf{1977}, vol.~13, 3, pp.~289--302.
	\item Also written by Mányu! Chronology? Meter?
\end{itemize}
\pause
\begin{center}
\begin{multicols}{2}
\scriptsize{
	Keṡī́ Agníṁ, keṡī́ viṣám,

	\vspace{\baselineskip}

	keṡī́ bibʰarti ródasī.

	Keṡī́ víṡvaṁ súar dr̥ṡé.

	Keṡī́ 'dáṁ jyótir ucyate.

	\vspace{\baselineskip}

	Múnayo vā́ta-raṡanāḥ

	piṡáṅgā vasate málā.

	Vā́tasy' ā́nu dʰrā́jiṁ yanti,

	yád devā́so ávikṣata.

	\columnbreak

	\textit{The long-haired (bears) Agní, the long-haired (bears) the waters,}

	\textit{the long-haired bears the Heavens and Earth.}

	\textit{The long-haired is all the light to see,}

	\textit{the long-haired is called this light.}

	\vspace{\baselineskip}

	\textit{The múni-s, with wind for reins,}

	\textit{wear (clothes) soiled tawny.}

	\textit{They follow the wind's motion}

	\textit{where the gods have entered.}
}
\end{multicols}
\end{center}
\end{frame}

\begin{frame} \frametitle{The Vedic múni-s: R̥V 10.136}
\begin{center}
\begin{multicols}{2}
\scriptsize{
	Únmaditā māúneyena

	vā́tā̃ ā́ tastʰimā vayám.

	Ṡárīr' éd asmā́kaṁ yūyám

	mártāso abʰí paṡyatʰa.

	\vspace{\baselineskip}

	Antárikṣeṇa patati

	víṡvā rūpā́ 'va cā́kaṡat.

	Múnir devásya--devasya

	sāúkr̥tyāya sákʰā hitáḥ.

	\vspace{\baselineskip}

	Vā́tasy' ā́ṡvo vāyóḥ sákʰā

	atʰ' o devé-'ṣito múniḥ.

	Ubʰāú samudrā́v ā́ kṣeti

	yáṡ ca pū́rva ut' ā́paraḥ.

	\columnbreak

	\textit{Enraptured by múni-ness}

	\textit{we have mounted the winds.}

	\textit{Our bodies indeed, O ye}

	\textit{mortals, look upon.}

	\vspace{\baselineskip}

	\textit{He flies in the air,}

	\textit{beholding every form.}

	\textit{The múni is of every god}

	\textit{made the friend for good work.}

	\vspace{\baselineskip}

	\textit{The wind's steed, Vāyú's friend,}

	\textit{is the god-moved múni.}

	\textit{He inhabits both the oceans}

	\textit{the eastern and the western alike.}
}
\end{multicols}
\end{center}
\end{frame}

\begin{frame} \frametitle{The Vedic múni-s: R̥V 10.136}
\begin{center}
\begin{multicols}{2}
\scriptsize{
	Apsarásāṁ Gandʰarvā́ṇām

	mr̥gā́ṇāṁ cáraṇe cáran

	keṡī́ kétasya viduā́n

	sákʰā svādúr madíntamaḥ.

	\vspace{\baselineskip}

	Vāyúr asmā úp’ āmantʰat.

	Pináṣṭi smā kunannamā́

	keṡī́ viṣásya pā́treṇa

	yád Rudréṇ’ ā́pibat sahá.
	
	\columnbreak

	\textit{The Apsarás-es', Gandʰarvá-s'}

	\textit{and beasts' path following,}

	\textit{the long-haired, knowing the will,}

	\textit{is a friend sweet and most delightful.}

	\vspace{\baselineskip}

	\textit{Vāyú has mixed (the drink) for him.}

	\textit{He pounds the inflexible}

	\textit{when the long-haired has drunk water}

	\textit{from the cup with Rudrá.}

}
\end{multicols}
\end{center}
\end{frame}

\begin{frame} \frametitle{What is Yóga?}
\begin{itemize}
	\item Werner: ``The \emph{keśins} as well as the \emph{śramaṇas} were not dissenters from the orthodox religion, but rather the representative of a different tradition.''
	\Subitem But what evidence is there for either position?
	\Subitem None, in my opinion
	\item Common pitfall in (especially western) Indology: contradiction vs.~synthesis
	\item See also: the Vrā́tya-s
\end{itemize}
\end{frame}

\begin{frame} \frametitle{Haṭʰa-yóga in the Vedá-s?}
\begin{itemize}
	\item Mudrā́-s? Ā́sana-s?
	\Subitem No.
	\item Breath control? ``The Science of Respiration and the Doctrine of the Bodily Winds in Ancient India.''
	\Subitem Zysk, K.~G. \emph{Journal of the American Oriental Society}, \textbf{1993}, vol.~113, 2, pp.~198--213.
	\item Atʰarva-vedá: strong focus on prāṇá
	\item But is this really Haṭʰa-yóga?
\end{itemize}
\end{frame}

\againframe{questions}

\end{document}
