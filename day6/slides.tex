\documentclass[pdf]{beamer}
\mode<presentation>{}
\usetheme{Rochester}
\usecolortheme{whale}
\beamertemplatenavigationsymbolsempty

\usepackage{fontspec}
\setmainfont[ItalicFont={brilli.ttf}, BoldFont={brillb.ttf}]{brill.ttf}
\usefonttheme{serif}
\usepackage{microtype}
\usepackage{multicol}

\newcommand{\Subitem}[1]{{\setlength\itemindent{12pt} \item[-] #1}}
\newcommand{\Subsubitem}[1]{{\setlength\itemindent{24pt} \item[○] #1}}
\newcommand{\Subsubsubitem}[1]{{\setlength\itemindent{36pt} \item[-] #1}}
\newcommand{\Subsubsubsubitem}[1]{{\setlength\itemindent{48pt} \item[○] #1}}

\title{Meeting 6: The Vedic pitch accent}
\subtitle{}
\author{Nikhil Surya Dwibhashyam, Rohan Pandey}
\date{3 April 2022}

\begin{document}

\frame{\titlepage}

\begin{frame} \frametitle{WhatsApp group}
\begin{center}
\href{https://chat.whatsapp.com/IXCQEkhfrcwHI7CNY8Fgat}{https://chat.whatsapp.com/IXCQEkhfrcwHI7CNY8Fgat}
\end{center}
\end{frame}

\begin{frame} \frametitle{Meeting agenda}
\begin{itemize}
	\item (Re)introduction to discussion group
	\item Today's meeting topic
	\item Free discussion
\end{itemize}
\end{frame}

\begin{frame} \frametitle{Why a Vedá discussion group?}
\begin{itemize}
	\item What is Vedic literature?
	\Subitem Sáṁhitā-s
	\Subitem Brā́hmaṇa-s
	\Subitem Others?
	\pause \item Vedá-s vs.~Vedā́nta \& Upaniṣád-s
	\Subitem Later monism (ádvāita) vs.~earlier dualism (dvāitá)
	\item Modern (religious, scholarly) fixation upon later literature
	\pause \item What is there to discuss?
	\Subitem History \& linguistics
	\Subitem Metaphysics of religion
	\Subitem Exegesis (e.g.~countering western narratives)
	\Subitem Moral principles
	\Subitem Nuances of Ṡrāutá ritual
\end{itemize}
\end{frame}

\begin{frame} \frametitle{Our plans}
\begin{itemize}
	\item Weekly meetings
	\item Different topic/aspect/angle discussed in depth every meeting
	\item Format:
	\Subitem Præsentations
	\Subitem Free-form discussions
	\item Eventually something more?
\end{itemize}
\end{frame}

\begin{frame} \frametitle{Let's introduce ourselves!}
\begin{itemize}
	\item Name
	\item School (if student)
	\item How you found this group
	\item Background/interest in Hinduism \& Vedic literature
\end{itemize}
\end{frame}

\begin{frame}[label=questions] \frametitle{Some questions to consider}
\begin{itemize}
	\item Why is the Vedic accent important?
	\item When, how, and why was it lost?
	\item How was the Vedic accent originally pronounced in speech?
	\item Is the saṁhitā-pā́ṭʰa really the original pronunciation of the Vedá-s? (No.)
	\item Can the Vedic accent be applied to later Sanskrit?
	\item We'll return to these at the end!
\end{itemize}
\end{frame}

\begin{frame} \frametitle{What is the Vedic accent?}
\begin{itemize}
	\item One syllable (or rarely two) takes an emphasis denoted by an acute accent
	\Subitem {\textit{devás}, \textit{devéna}}
	\Subitem {\textit{pā́t}, \textit{padā́}}
	\Subitem {\textit{kártavāí}, \textit{ráthas-páti}}
	\item Sometimes determined by rules (e.g.~when derived from verbal root); sometimes not!
	\Subitem For basic lemmata: must be memorized
	\item ``Independent svaritá''?
\end{itemize}
\end{frame}

\begin{frame} \frametitle{Philology}
\begin{itemize}
	\item Vedic accent and (Indo-Iranian, Indo-European) reconstruction
	\Subitem Much better than Ancient Greek!
	\item Best source of accent assignment
	\Subitem Contributes to many sound changes in other families
\end{itemize}
\end{frame}

\begin{frame} \frametitle{The tradition of different schools}
\begin{itemize}
	\item R̥V tradition considered ``standard'' even for many non-R̥V hymns
	\item Basic pattern: low preceding accented syllable, high following
	\item Blocked by another accented syllable nearby
	\item Totally breaks meter, particularly with svaritá-s
	\item Long svaritá: R̥V vs.~KYV
\end{itemize}
\end{frame}

\begin{frame} \frametitle{Restoring the accent in Classical}
\begin{itemize}
	\item Surprising number of words straightforward to reconstruct
	\item Otherwise: etymological reconstruction
	\Subitem Following standard rules from inference or Pāṇini
	\item Phonological reconstruction?
	\Subitem Not ideal: not entirely clear how syllable stress worked in Classical
\end{itemize}
\end{frame}

\begin{frame} \frametitle{Why is the accent important?}
\begin{itemize}
	\item Interpretation of precise meanings
	\Subitem One example: \textit{ṡiṡná-deva}
	\Subitem Clearly bahú-vrīhi (exocentric)
	\Subitem Is it \textit{déva} or \textit{devá}? Scholars differ.
	\item Proper recitation technique
	\item Construction of Vedic identity
	\Subitem The accent is perhaps the clearest delineation between the Vedic and non-Vedic eras.
	\item Others?
\end{itemize}
\end{frame}

\againframe{questions}

\end{document}
