\documentclass[pdf]{beamer}
\mode<presentation>{}
\usetheme{Rochester}
\usecolortheme{whale}
\beamertemplatenavigationsymbolsempty

\usepackage{fontspec}
\setmainfont[ItalicFont={brilli.ttf}, BoldFont={brillb.ttf}]{brill.ttf}
\usefonttheme{serif}
\usepackage{microtype}
\usepackage{multicol}

\newcommand{\Subitem}[1]{{\setlength\itemindent{12pt} \item[-] #1}}
\newcommand{\Subsubitem}[1]{{\setlength\itemindent{24pt} \item[○] #1}}

\title{Meeting 3: The hero of three wide steps}
\subtitle{Víṣṇu in the Vedá-s and the Purāṇá-s}
\author{Nikhil Surya Dwibhashyam, Rohan Pandey}
\date{6 March 2022}

\begin{document}

\frame{\titlepage}

\begin{frame} \frametitle{Meeting agenda}
\begin{itemize}
	\item (Re)introduction to discussion group
	\item Today's meeting topic
	\item Free discussion
\end{itemize}
\end{frame}

\begin{frame} \frametitle{Why a Vedá discussion group?}
\begin{itemize}
	\item What is Vedic literature?
	\Subitem Sáṁhitā-s
	\Subitem Brā́hmaṇa-s
	\Subitem Others?
	\pause \item Vedá-s vs.~Vedā́nta \& Upaniṣád-s
	\Subitem Later monism (ádvāita) vs.~earlier dualism (dvāitá)
	\item Modern (religious, scholarly) fixation upon later literature
	\pause \item What is there to discuss?
	\Subitem History \& linguistics
	\Subitem Metaphysics of religion
	\Subitem Exegesis (e.g.~countering western narratives)
	\Subitem Moral principles
	\Subitem Nuances of Ṡrāutá ritual
\end{itemize}
\end{frame}

\begin{frame} \frametitle{Our plans}
\begin{itemize}
	\item Weekly meetings
	\item Different topic/aspect/angle discussed in depth every meeting
	\item Format:
	\Subitem Præsentations
	\Subitem Free-form discussions
	\item Eventually something more?
\end{itemize}
\end{frame}

\begin{frame} \frametitle{Let's introduce ourselves!}
\begin{itemize}
	\item Name
	\item School (if student)
	\item How you found this group
	\item Background/interest in Hinduism \& Vedic literature
\end{itemize}
\end{frame}

\begin{frame}[label=questions] \frametitle{Some questions to consider}
\begin{itemize}
	\item What are the three steps of Víṣṇu? their significance?
	\item What is the role of Víṣṇu in the Vedic pantheon? What similarities are there to later religion?
	\item To what extent does Vaiṣṇavá have a basis in the Vedá-s?
	\item What predecessors of the classic Avatārá-s can be seen in much earlier literature?
	\item We'll return to these at the end!
\end{itemize}
\end{frame}

\begin{frame} \frametitle{Víṣṇu: an etymology in English}
\begin{itemize}
	\item Many gods' names originally nominals (e.g.~Ṡivá), especially agents (e.g.~Savitŕ̥)
	\item -nu (here with conditioned retroflex): agentive
	\Subitem e.g.~$\surd$\textit{dʰr̥ṣ} ``to dare'' $\rightarrow$ \textit{dʰr̥ṣṇú} ``brave''
	\item Víṣṇu: one who does $\surd$\textit{viṣ}
\end{itemize}

\begin{block} {$\surd$\textit{viṣ} (MW on R̥V, ṠB)}
\begin{itemize}
	\item to be active, act, work, do, perform
	\item to be quick, speed, run, flow
	\item to work as a servant, serve
	\item to have done with i.e.~overcome, subdue, rule
\end{itemize}
\end{block}
\end{frame}

\begin{frame} \frametitle{Víṣṇu: an etymology in German}
\begin{itemize}
	\item Emphasis on conquest:
\end{itemize}

\begin{block} {$\surd$\textit{viṣ} (Graßmann)}
\begin{enumerate}
  \setcounter{enumi}{4}
  \item feindlich ergreifen [A.], bewältigen [A.]
\end{enumerate}
\end{block}

\begin{itemize}
	\item but also unity:
\end{itemize}

\begin{block} {$\surd$\textit{viṣ} (Graßmann)}
\begin{enumerate}
  \setcounter{enumi}{9}
  \item sich vereinigen mit [I.]
\end{enumerate}
\end{block}

\begin{itemize}
	\item Overall:
\end{itemize}

\begin{block} {\textit{víṣṇu} (Graßmann)}
\begin{enumerate}
  \item wirksam
\end{enumerate}
\end{block}
\end{frame}

\begin{frame} \frametitle{Víṣṇu in the R̥g-vedá}
\begin{itemize}
	\item Hymns addressed to Víṣṇu: 1.154, 1.156, 7.100
	\item Hymns addressed to Víṣṇu and Índra: 1.155, 7.99
	\item Minor god? Not necessarily.
\end{itemize}

\begin{center}
	Víṣṇor nú kaṁ • vīríāṇi prá vocam,

	yáḥ pā́rtʰivāni • vi mamé rájāṁsi,

	yó áskabʰāyad • úttaraṁ sadʰá-stʰam,

	vi-cakramāṇás • traya-dʰó 'ru-gāyáḥ.

	\vspace{\baselineskip}
	
	\textit{Víṣṇu's heroic deeds I shall proclaim,}

	\textit{who measured out the Earthly and the Heavenly (regions),}

	\textit{(and) who propped up the higher abode,}

	\textit{striding thrice with wide steps.}

	\vspace{\baselineskip}

	---R̥V 1.154.1
\end{center}
\end{frame}

\begin{frame} \frametitle{Víṣṇu in the R̥g-vedá}
\begin{itemize}
	\item Common attributes:
	\Subitem ``Thrice-going'', ``wide-going'' (everywhere)
	\Subitem ``A youth who is no child'' (1.155.6)
	\Subitem ``Upholding the laws'' (1.22.18)
	\Subitem ``Generating Sū́rya, Uṣás, and Agní'' (7.99.4)
	\Subitem ``slaying Vr̥trá'' (4.18.11), ``conquering the Dásyu'' (7.99.4)
\end{itemize}
\end{frame}

\begin{frame} \frametitle{The development of Vaiṣṇavá}
\begin{itemize}
	\item Origin in Brā́hmaṇa literature
	\Subitem Also origins of many Avatārá-s!
	\item Development as supreme being
	\Subitem {\emph{beyond} godhood}
\end{itemize}

\begin{center}
	Vadanti tat tattvavidas

	tattvaṁ yaj jñānam advayam:

	Brahme 'ti Paramātme 'ti

	Bʰagavān iti ṡabdyate.

	\vspace{\baselineskip}

	\textit{The knowers of truth speak}

	\textit{the truth that is non-dual knowledge:}

	\textit{As Brahman, as Paramātman,}

	\textit{as Bʰagavān it is called.}

	\vspace{\baselineskip}

	---BʰP 1.2.11
\end{center}
\end{frame}

\begin{frame} \frametitle{The development of Vaiṣṇavá}
\begin{itemize}
	\item Caste politics? Kṣatríya-s vs.~Brāhmaṇá-s
	\Subitem Development of ``Brahmanical'' qualities in Víṣṇu
	\Subitem Power struggle with unique outcome
	\item Humiliation, denigration of Índra (becomes common theme)
	\Subitem New rivalry between Índra and Víṣṇu!
\end{itemize}
\end{frame}

\againframe{questions}

\end{document}
