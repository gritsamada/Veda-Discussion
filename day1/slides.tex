\documentclass[pdf]{beamer}
\mode<presentation>{}
\usetheme{Rochester}
\usecolortheme{whale}
\beamertemplatenavigationsymbolsempty

\usepackage{fontspec}
\setmainfont[ItalicFont={brilli.ttf}, BoldFont={brillb.ttf}]{brill.ttf}
\usefonttheme{serif}
\usepackage{microtype}
\usepackage{multicol}

\newcommand{\Subitem}[1]{{\setlength\itemindent{12pt} \item[-] #1}}
\newcommand{\Subsubitem}[1]{{\setlength\itemindent{24pt} \item[○] #1}}

\title{Meeting 1: Vratá-s, Satyá, \& Mitrá--Váruṇa}
\subtitle{The Vedic conception of oath-keeping}
\author{Nikhil Surya Dwibhashyam, Rohan Pandey}
\date{20 February 2022}

\begin{document}

\frame{\titlepage}

\begin{frame} \frametitle{Meeting agenda}
\begin{itemize}
	\item Introduction to discussion group
	\item Today's meeting topic
	\item Free discussion
\end{itemize}
\end{frame}

\begin{frame} \frametitle{Why a Vedá discussion group?}
\begin{itemize}
	\item What is Vedic literature?
	\Subitem Sáṁhitā-s
	\Subitem Brā́hmaṇa-s
	\Subitem Others?
	\pause \item Vedá-s vs.~Vedā́nta \& Upaniṣád-s
	\Subitem Later monism (ádvāita) vs.~earlier dualism (dvāitá)
	\item Modern (religious, scholarly) fixation upon later literature
	\pause \item What is there to discuss?
	\Subitem History \& linguistics
	\Subitem Metaphysics of religion
	\Subitem Exegesis (e.g.~countering western narratives)
	\Subitem Moral principles
	\Subitem Nuances of Ṡrāutá ritual
\end{itemize}
\end{frame}

\begin{frame} \frametitle{Our plans}
\begin{itemize}
	\item Weekly meetings
	\item Different topic/aspect/angle discussed in depth every meeting
	\item Format:
	\Subitem Presentations
	\Subitem Free-form discussions
	\item Eventually something more?
\end{itemize}
\end{frame}

\begin{frame} \frametitle{Let's introduce ourselves!}
\begin{itemize}
	\item Name
	\item School
	\item Background/interest in Hinduism \& Vedic literature
\end{itemize}
\end{frame}

\begin{frame} \frametitle{Today's topic: Vratá-s, Satyá, \& Mitrá--Váruṇa}
\begin{block} {vratá n.~(MW on R̥V)}
\begin{itemize}
	\item will, command, law, ordinance, rule
	\item sphere of action, function, mode or manner of life, conduct, manner, usage, custom
	\item a religious vow or practice, any pious observance, meritorious act of devotion or austerity, solemn vow, rule, holy practice
\end{itemize}
\end{block}
\pause
\begin{block} {satyá n.~(MW)}
\begin{itemize}
	\item truth, reality (R̥V \&c)
	\item a solemn asseveration, vow, promise, oath (AV \&c)
\end{itemize}
\end{block}
\end{frame}

\begin{frame}[label=questions] \frametitle{Some questions to consider}
\begin{itemize}
	\item What is uniquely Vedic about the vratá?
	\item What are the religious \& social consequences of breaking an oath?
	\item Why is the oath so fundamentally important to the Ā́rya?
	\item What oaths must an Ā́rya make?
	\item What are the implications of strict oath-keeping on our personal lives?~on social organization?
	\item We'll return to these at the end!
\end{itemize}
\end{frame}

\begin{frame} \frametitle{The best place to start}
\begin{itemize}
	\item Usually a concordance!
	\item e.g.~searching \texttt{.*vrat.*} in Graßmann's dictionary
	\Subitem {\textit{ánu-vrata}, \textit{anyá-vrata}, \textit{avratá}, \ldots, \textit{su-vratá}}
	\Subitem vŕ̥ddʰi-ed form also useful: \texttt{.*vrāt.*}
	\item VedaWeb also a good resource
	\item What do we notice from the words' contexts?
\end{itemize}
\end{frame}

\begin{frame} \frametitle{Are oaths fundamental to the Ā́rya identity?}
\begin{itemize}
	\item \textit{Ā́rya}: a vŕ̥ddʰi-ed form of \textit{aryá} (Graßmann: treu, ergeben, fromm)?
	\pause \item \textit{Aryá} probably from \textit{arí} (MW: faithul, devoted, pious), not to be confused with \textit{ári} (enemy!)
	\pause \item Loyalty/faithfulness (quā oath-keeping) as a distinguishing trait of ethno-religious Ā́rya identity?
\end{itemize}
\end{frame}

\begin{frame} \frametitle{Are oaths fundamental to the Ā́rya identity?}
\begin{center}
	Ánu-vratāya • randʰáyann ápa-vratān

	ābʰū́bʰir Índraḥ • ṡnatʰáyann ánābʰuvaḥ.

	\vspace{12pt}

	\textit{Subduing the oath-breaking for the oath-following,}

	\textit{Índra destroys the strengthless by the strong.}

	\vspace{12pt}

 	---R̥V 1.51.9ab
\end{center}

\begin{itemize}
	\item 1.51.8 makes clear: former are Dásyu-s, latter are Ā́rya-s
	\item Herrenmoral?~but also!~piousness, loyalty
\end{itemize}
\end{frame}

\begin{frame} \frametitle{Are oaths fundamental to the Ā́rya identity?}
\begin{itemize}
	\item The enemies of the Vedic religion are distinguished most of all by their oath-breaking/lawlessness.
	\item Famous verse on the Dásyu:
\end{itemize}
\begin{center}
	Akarmā́ Dásyur • abʰí no amantúḥ

	anyá-vrato ámānuṣaḥ.

	Tuáṁ tásya amitrahan

	vádʰar Dāsásya dambʰaya!

	\vspace{12pt}

	\textit{All around us is the Dásyu, wicked, mindless,}

	\textit{alien of oath, inhuman.}

	\textit{O thou foe-slayer,}

	\textit{destroy that Dāsá's weapon!}

	\vspace{12pt}

 	---R̥V 10.22.8
\end{center}
\end{frame}

\begin{frame} \frametitle{The gods \& dæmons, too, are bound by vratá-s.}\
\begin{center}
	Ná yásya Índro • Váruṇo ná Mitráḥ

	vratám Aryamā́ • ná minánti Rudráḥ

	ná árātayas, • tám idáṁ suastí

	hué deváṁ • Savitā́raṁ námobʰiḥ.

	\vspace{12pt}

	\textit{Whose oath neither Índra nor Váruṇa nor Mitrá}

	\textit{nor Aryamán nor Rudrá violates,}

	\textit{nor dæmons: for welfare, that very}

	\textit{god Savitŕ̥ I invoke with salutations.}

	\vspace{12pt}

 	---R̥V 2.38.9
\end{center}
\end{frame}

\begin{frame} \frametitle{Vedic duality: avratá \& asatyá}
\begin{itemize}
	\item Álpʰa-privative often just as informative as the word itself
	\item We already saw \textit{avratá}, as well as:
	\Subitem {\textit{ápa-vrata}, \textit{anyá-vrata}}
	\item Asatyá (once in R̥V but illuminating):
\end{itemize}
\end{frame}

\begin{frame} \frametitle{Vedic duality: avratá \& asatyá}
\begin{center}
\begin{multicols}{2}
\small{
	Prá tā̃́ Agnír • babʰasat tigmá-jambʰaḥ

	tápiṣṭʰena • ṡocíṣā yáḥ surā́dʰāḥ,

	prá yé minánti • Váruṇasya dʰā́ma

	priyā́ Mitrásya • cétato dʰruvā́ṇi.

	\vspace{12pt}

	Abʰrātáro ná • yóṣaṇo viántaḥ

	pati-rípo ná • jánayo durévāḥ

	pāpā́saḥ sánto • Anr̥tā́ asatyā́ḥ

	idám padám • ajanatā gabʰīrám.

	\vspace{12pt}

	\textit{May sharp-toothed generous Agní}

	\textit{consume with the hottest flame}

	\textit{those who violate the laws of Váruṇa,}

	\textit{the dear steadfast (laws) of wise Mitrá.}

	\vspace{12pt}

	\textit{Like brotherless maidens, straying,}

	\textit{like husband-betraying women, wicked,}

	\textit{those who are sinful, against R̥tá, untrue,}

	\textit{have brought forth this deep place [hell?].}
}
\end{multicols}
 ---R̥V 4.5.4--5
\end{center}
\end{frame}

\begin{frame} \frametitle{Vedic duality: Mitrá--Váruṇa}
\begin{itemize}
	\item Mitrá and Váruṇa as a dvaṁ-dvá: guardians of divine law
\end{itemize}
\begin{center}
	Dʰármaṇā Mitrā--•Varuṇā vipaṡcitā

	vratā́ rakṣetʰe • ásurasya māyáyā.

	R̥téna víṡvam • bʰúvanaṁ ví rājatʰaḥ.

	Sū́ryam ā́ dʰattʰo • diví cítriaṁ rátʰam.

	\vspace{12pt}

	\textit{O wisdom-inspired Mitrá--Váruṇa, with the law}

	\textit{and with the power of the divine ye guard oaths.}

	\textit{Ye govern all existence by R̥tá.}

	\textit{Ye set the Sun in heaven as a bright chariot.}

	\vspace{12pt}

 	---R̥V 5.63.7
\end{center}
\end{frame}

\begin{frame} \frametitle{The structure of the Vedic religion}
(My own theory!)
\begin{itemize}
	\item R̥tá 
	\Subitem {Dʰárman ($\surd$\textit{dʰr̥})}
	\Subsubitem Vratá-s
	\Subitem {Satyá ($\surd$\textit{as})}
	\Subsubitem Vratá-s
\end{itemize}
\begin{itemize}
	\item {Savitŕ̥ (prīmum movēns, $\surd$\textit{sū})}
	\Subitem Mitrá--Váruṇa
	\Subsubitem Sū́rya
	\item Índra (king), Agní (commander)
	\Subitem Víṡve devā́s
	\Subsubitem Ā́rya-s \&c
\end{itemize}
\end{frame}

\begin{frame} \frametitle{What happens to the avratá-s?}
\begin{itemize}
	\item The vratá is a choice: derived from $\surd$\textit{vr̥} ``choose''
	\Subitem One can, and many do, choose to break oaths. What then?
	\item Natural consequences:
\end{itemize}
\begin{center}
	Su-gáḥ pantʰā anr̥kṣaráḥ

	Ā́dityāsa R̥táṁ yaté.

	N’ ā́tr’ āva-kʰādáv asti vaḥ.

	\vspace{12pt}

	\textit{Easy and thornless is the path,}

	\textit{O sons of Áditi, for the follower of R̥tá.}

	\textit{Not then is there cause to anger for ye.}

	\vspace{12pt}

 	---R̥V 1.41.4
\end{center}
\begin{itemize}
	\item Also hints at wrath of gods: earthly?~hell?
	\Subitem Hard to tell
\end{itemize}
\end{frame}

\begin{frame} \frametitle{Vrā́tya-s, Brāhmaṇá-s \& the vratá: social organization}
\begin{block} {vrā́ta m.~(MW on R̥V)}
\begin{itemize}
	\item a multitude, flock, assemblage, troop, swarm, group, host, association, guild
\end{itemize}
\end{block}
\begin{itemize}
	\item The Vrā́tya-s?
	\item Priestly oaths to serve god: the basis for the later \textit{vrata} concept
	\item Oaths as a means of binding people(s) together
	\Subitem Husband and wife
	\Subitem Teacher and student
	\Subitem Different Aryan tribes
\end{itemize}
\end{frame}

\begin{frame} \frametitle{The vratá as a profession}
\begin{center}
	Nānānáṁ vā́ u no dʰíyaḥ

	ví vratā́ni jánānaam.

	Tákṣā riṣṭáṁ rutám bʰiṣák

	brahmā́ sunvántam iccʰati.

	\vspace{12pt}

	\textit{Diverse indeed are our thoughts,}

	\textit{and so the oaths of men.}

	\textit{The builder seeks the broken, the doctor the injured,}

	\textit{the priest the worshiper.}

	\vspace{12pt}

 	---R̥V 9.112.1a--d
\end{center}
\end{frame}

\begin{frame} \frametitle{The vrata in later literature}
\begin{itemize}
	\item More ritual-focused (ironically!)
	\item Vows of fasting, chastity, silence, etc.
	\item Central meaning still retained, e.g.:
\end{itemize}
\begin{center}
	Su-bʰagā bʰoga-saṁpannā

	yajña-patnī sv-anu-vratā.

	\vspace{12pt}

	\textit{Fortunate, endowed with joy,}

	\textit{(be) the wife at the sacrifice, keeping well (thy) oaths.}

	\vspace{12pt}

 	---MBʰ 1.191.7cd
\end{center}
\end{frame}

\againframe{questions}

\end{document}
