\documentclass[pdf]{beamer}
\mode<presentation>{}
\usetheme{Rochester}
\usecolortheme{whale}
\beamertemplatenavigationsymbolsempty

\usepackage{fontspec}
\setmainfont[ItalicFont={brilli.ttf}, BoldFont={brillb.ttf}]{brill.ttf}
\usefonttheme{serif}
\usepackage{microtype}
\usepackage{multicol}

\newcommand{\Subitem}[1]{{\setlength\itemindent{12pt} \item[-] #1}}
\newcommand{\Subsubitem}[1]{{\setlength\itemindent{24pt} \item[○] #1}}
\newcommand{\Subsubsubitem}[1]{{\setlength\itemindent{36pt} \item[-] #1}}
\newcommand{\Subsubsubsubitem}[1]{{\setlength\itemindent{48pt} \item[○] #1}}

\title{Meeting 5: The Aryan fusion}
\subtitle{Examining the non-IE substratum in Vedic}
\author{Nikhil Surya Dwibhashyam, Rohan Pandey}
\date{20 March 2022}

\begin{document}

\frame{\titlepage}

\begin{frame} \frametitle{WhatsApp group}
\begin{center}
\href{https://chat.whatsapp.com/IXCQEkhfrcwHI7CNY8Fgat}{https://chat.whatsapp.com/IXCQEkhfrcwHI7CNY8Fgat}
\end{center}
\end{frame}

\begin{frame} \frametitle{Meeting agenda}
\begin{itemize}
	\item (Re)introduction to discussion group
	\item Today's meeting topic
	\item Free discussion
\end{itemize}
\end{frame}

\begin{frame} \frametitle{Why a Vedá discussion group?}
\begin{itemize}
	\item What is Vedic literature?
	\Subitem Sáṁhitā-s
	\Subitem Brā́hmaṇa-s
	\Subitem Others?
	\pause \item Vedá-s vs.~Vedā́nta \& Upaniṣád-s
	\Subitem Later monism (ádvāita) vs.~earlier dualism (dvāitá)
	\item Modern (religious, scholarly) fixation upon later literature
	\pause \item What is there to discuss?
	\Subitem History \& linguistics
	\Subitem Metaphysics of religion
	\Subitem Exegesis (e.g.~countering western narratives)
	\Subitem Moral principles
	\Subitem Nuances of Ṡrāutá ritual
\end{itemize}
\end{frame}

\begin{frame} \frametitle{Our plans}
\begin{itemize}
	\item Weekly meetings
	\item Different topic/aspect/angle discussed in depth every meeting
	\item Format:
	\Subitem Præsentations
	\Subitem Free-form discussions
	\item Eventually something more?
\end{itemize}
\end{frame}

\begin{frame} \frametitle{Let's introduce ourselves!}
\begin{itemize}
	\item Name
	\item School (if student)
	\item How you found this group
	\item Background/interest in Hinduism \& Vedic literature
\end{itemize}
\end{frame}

\begin{frame}[label=questions] \frametitle{Some questions to consider}
\begin{itemize}
	\item What types of loanwords are present in the Vedic period?
	\Subitem from which language families?
	\item How extensive is loanwords adoption in Vedic?
	\item Why and how were these loanwords adopted from (ostensible) enemies of the Aryans? Were they enemies?
	\item How much evidence is there that these candidates are in fact loanwords (vs., for example, innovations)?
	\item Metaphysically, is the adoption of these loanwords contradictory to the Vedic religion?
	\item What sort of substrata are present besides loanwords?
\end{itemize}
\end{frame}

\begin{frame} \frametitle{What is a substratum?}
\begin{itemize}
	\item Usually result of power imbalance
	\Subitem e.g.~conquest
	\item Different types:
	\Subitem Religious--ideological
	\Subitem Social structures
	\Subitem Linguistic
	\item Substrate vs.~superstrate: e.g.~Mitanni
\end{itemize}
\end{frame}

\begin{frame} \frametitle{Religious substratum}
\begin{itemize}
	\item Aryan superstrate influence on Dravidian (obvious)
	\Subitem on others?
	\item Dravidian influence (possibly = IVC influence)?
	\Subitem Unlikely: conflict between traditions
	\Subsubitem {e.g.~\textit{ṡiṡná-deva}}
	\item Other candidates?
	\Subitem BMAC
	\Subitem (Para-)Munda
\end{itemize}
\end{frame}

\begin{frame} \frametitle{Religious substratum}
\begin{itemize}
	\item Munda?
	\Subitem {Dæmons, e.g.~\textit{kimīdín}, \textit{kárañja}}
	\Subitem {\textit{kīstá}}
	\item BMAC?
	\Subitem Sóma
	\Subsubitem Archæological ``evidence''
	\Subitem Índra: total speculation (like a lot of this)
\end{itemize}
\end{frame}

\begin{frame} \frametitle{General linguistic substratum}
\begin{itemize}
	\item Very early: Central Asian
	\Subitem Animals (camel, lion), technologies (bricks, channel, lute)
	\item BMAC
	\Subitem Methodology: commonality between Iranian and Aryan
	\item Dravidian
	\Subitem Witzel (1999): not as much as originally thought
	\Subitem Agricultural (threshing floor, etc.)
	\item Munda or Para-Munda
	\Subitem kV prefix
	\item ``Language X''
	\Subitem Gemination
\end{itemize}
\end{frame}

\againframe{questions}

\end{document}
