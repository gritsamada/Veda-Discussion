\documentclass[pdf]{beamer}
\mode<presentation>{}
\usetheme{Rochester}
\usecolortheme{whale}
\beamertemplatenavigationsymbolsempty

\usepackage{fontspec}
\setmainfont[ItalicFont={brilli.ttf}, BoldFont={brillb.ttf}]{brill.ttf}
\usefonttheme{serif}
\usepackage{microtype}

\newcommand{\Subitem}[1]{{\setlength\itemindent{12pt} \item[-] #1}}

\title{Meeting 1: Vratá-s, Satyá, \& Mitrá--Váruṇa}
\subtitle{The Vedic conception of oath-keeping}
\author{Nikhil Surya Dwibhashyam, Rohan Pandey}
\date{20 February 2022}

\begin{document}

\frame{\titlepage}

\begin{frame} \frametitle{Meeting agenda}
\begin{itemize}
	\item Introduction to discussion group
	\item Today's meeting topic
	\item Free discussion
\end{itemize}
\end{frame}

\begin{frame} \frametitle{Why a Vedá discussion group?}
\begin{itemize}
	\item What is Vedic literature?
	\Subitem Sáṁhitā-s
	\Subitem Brā́hmaṇa-s
	\Subitem Others?
	\pause \item Vedá-s vs.~Vedā́nta \& Upaniṣád-s
	\Subitem Later monism (ádvāita) vs.~earlier dualism (dvāitá)
	\item Modern fixation upon later literature
	\pause \item What is there to discuss?
	\Subitem History \& linguistics
	\Subitem Metaphysics of religion
	\Subitem Moral principles
	\Subitem Nuances of Ṡrāutá ritual
\end{itemize}
\end{frame}

\begin{frame} \frametitle{Our plans}
\begin{itemize}
	\item Weekly meetings
	\item Different topic/aspect/angle discussed in depth every meeting
	\item Presentations
	\item Free-form discussions
	\item Eventually something more?
\end{itemize}
\end{frame}

\begin{frame} \frametitle{Today's topic: Vratá-s, Satyá, \& Mitrá--Váruṇa}
\begin{block} {vratá n.~(MW on R̥V)}
\begin{itemize}
	\item will, command, law, ordinance, rule
	\item sphere of action, function, mode or manner of life, conduct, manner, usage, custom
	\item a religious vow or practice, any pious observance, meritorious act of devotion or austerity, solemn vow, rule, holy practice
\end{itemize}
\end{block}
\pause
\begin{block} {satyá n.~(MW)}
\begin{itemize}
	\item truth, reality (R̥V \&c)
	\item a solemn asseveration, vow, promise, oath (AV \&c)
\end{itemize}
\end{block}
\end{frame}

\begin{frame} \frametitle{Some questions to consider}
\begin{itemize}
	\item What is uniquely Vedic about the vratá?
	\item What are the religious \& social consequences of breaking an oath?
	\item Why is the oath so fundamentally important to the Ā́rya?
	\item What oaths must an Ā́rya make?
	\item What are the implications of strict oath-keeping on our personal lives?
	\Subitem on social organization?
\end{itemize}
\end{frame}

\begin{frame} \frametitle{The best place to start}
\begin{itemize}
	\item Usually a collocation!
	\item e.g.~searching \texttt{.*vrat.*} in Graßmann's dictionary
	\Subitem {\textit{ánu-vrata}, \textit{anyá-vrata}, \textit{avratá}, \ldots, \textit{su-vratá}}
	\item What do we notice from the words' contexts?
\end{itemize}
\end{frame}

\begin{frame} \frametitle{Are oaths fundamental to the Ā́rya identity?}
\begin{itemize}
	\item \textit{Ā́rya}: a vŕ̥ddʰi-ed form of \textit{aryá} (Graßmann: treu, ergeben, fromm)?
	\pause \item \textit{Aryá} probably from \textit{arí} (MW: faithul, devoted, pious), not to be confused with \textit{ári} (enemy!)
	\pause \item Loyalty/faithfulness (quā oath-keeping) as a distinguishing trait of ethnoreligious Ā́rya identity?
\end{itemize}
\end{frame}

\begin{frame} \frametitle{Are oaths fundamental to the Ā́rya identity?}
\begin{center}
	Ánu-vratāya • randʰáyann ápa-vratān

	ābʰū́bʰir Índraḥ • ṡnatʰáyann ánābʰuvaḥ.

	\vspace{12pt}

	\textit{Subduing the oath-breaking for the oath-following,}

	\textit{Índra destroys the strengthless by the strong.}

	\vspace{12pt}

 	---R̥V 1.51.9ab
\end{center}

\begin{itemize}
	\item 1.51.8 makes clear: former are Dásyu-s, latter are Ā́rya-s
	\item Herrenmoral? but also! piousness, loyalty
\end{itemize}
\end{frame}

\begin{frame} \frametitle{Vedic duality: avratá \& asatyá}
%variant words (ápa-vrata)
%What happens to those who break oaths?
%Introduce Mitrá--Váruṇa
\end{frame}

\begin{frame} \frametitle{Vedic duality: Mitrá--Váruṇa}
\end{frame}

\begin{frame} \frametitle{What happens to the avratá-s?}
%punishment on Earth
%in afterlife?
\end{frame}

\begin{frame} \frametitle{Vrā́tya-s, Brāhmaṇá-s \& the vratá}
\end{frame}

\end{document}
