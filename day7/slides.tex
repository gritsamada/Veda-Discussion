\documentclass[pdf]{beamer}
\mode<presentation>{}
\usetheme{Rochester}
\usecolortheme{whale}
\beamertemplatenavigationsymbolsempty

\usepackage{fontspec}
\setmainfont[ItalicFont={brilli.ttf}, BoldFont={brillb.ttf}]{brill.ttf}
\usefonttheme{serif}
\usepackage{microtype}
\usepackage{multicol}

\newcommand{\Subitem}[1]{{\setlength\itemindent{12pt} \item[-] #1}}
\newcommand{\Subsubitem}[1]{{\setlength\itemindent{24pt} \item[○] #1}}
\newcommand{\Subsubsubitem}[1]{{\setlength\itemindent{36pt} \item[-] #1}}
\newcommand{\Subsubsubsubitem}[1]{{\setlength\itemindent{48pt} \item[○] #1}}

\title{Meeting 7: Áhiṁsā}
\subtitle{Nonviolence and vegetarianism in earlier and later Hinduism}
\author{Nikhil Surya Dwibhashyam, Rohan Pandey}
\date{10 April 2022}

\begin{document}

\frame{\titlepage}

\begin{frame} \frametitle{WhatsApp group}
\begin{center}
\href{https://chat.whatsapp.com/IXCQEkhfrcwHI7CNY8Fgat}{https://chat.whatsapp.com/IXCQEkhfrcwHI7CNY8Fgat}
\end{center}
\end{frame}

\begin{frame} \frametitle{Meeting agenda}
\begin{itemize}
	\item (Re)introduction to discussion group
	\item Today's meeting topic
	\item Free discussion
\end{itemize}
\end{frame}

\begin{frame} \frametitle{Why a Vedá discussion group?}
\begin{itemize}
	\item What is Vedic literature?
	\Subitem Sáṁhitā-s
	\Subitem Brā́hmaṇa-s
	\Subitem Others?
	\pause \item Vedá-s vs.~Vedā́nta \& Upaniṣád-s
	\Subitem Later monism (ádvāita) vs.~earlier dualism (dvāitá)
	\item Modern (religious, scholarly) fixation upon later literature
	\pause \item What is there to discuss?
	\Subitem History \& linguistics
	\Subitem Metaphysics of religion
	\Subitem Exegesis (e.g.~countering western narratives)
	\Subitem Moral principles
	\Subitem Nuances of Ṡrāutá ritual
\end{itemize}
\end{frame}

\begin{frame} \frametitle{Our plans}
\begin{itemize}
	\item Weekly meetings
	\item Different topic/aspect/angle discussed in depth every meeting
	\item Format:
	\Subitem Præsentations
	\Subitem Free-form discussions
	\item Eventually something more?
\end{itemize}
\end{frame}

\begin{frame} \frametitle{Let's introduce ourselves!}
\begin{itemize}
	\item Name
	\item School (if student)
	\item How you found this group
	\item Background/interest in Hinduism \& Vedic literature
\end{itemize}
\end{frame}

\begin{frame}\frametitle{}
\begin{itemize}
	\item We'll return to these at the end!
\end{itemize}
\end{frame}

\begin{frame}{áhiṁsā}
    \begin{itemize} 
    \item A negative tatpuruṣa compound (2.2.6). From the verbal root हिसि हिंसायाम् (5/A) with suffix अङ् (3.3.104). \textbf{feminine} action noun with suffix आ from 	4.1.4.
    \item In post-Vedic literature she is the wife of Dharma and also later on development of the Shākta tradition. 
    \item AiG: Wackernagel (II:245-246) puts in the category of abstract form from "nichtabgeleiteten Verben" and cites SB and AB.
    \item TITUS search provides the word frequently occuring in the Dharmasūtras(DhS) particularly of the Black Yajur-Veda (Vaisnava VDhS: Verse 67, 68 Baudhayana 	BDhS) and less in RV DhS (Vasistha VasDhs- 4,4; 30,8  ):
    
	VDhs Verse: 67 
	Halfverse: a    yā veda-vihitā hiṃsā niyatā_asmiṃś cara-acare /
	Halfverse: c    ahiṃsām eva tāṃ vidyād\ vedād dʰarmo hi nirbabʰau\ //

	VDhs Verse: 68 
	Halfverse: a    yo_ahiṃsakāni bʰūtāni hinasty\ ātma-sukʰa-iccʰayā /
	Halfverse: c    sa jīvaṃś ca mr̥taś ca_eva na kva-cit sukʰam edʰate\ //
	
	VasDhs Verse: 4,4 
	Sentence: a    sarveṣāṃ satyam a-krodʰo dānam ahiṃsā prajananaṃ ca \4,4\.
	
	VasDhs Verse: 30,8
	Sentence: a    dʰyāna-agniḥ satya-upacayanaṃ kṣānty-āhutiḥ
	Sentence: b    sruvaṃ hrīḥ puroḍāśam ahiṃsā saṃtoṣo,
	Sentence: c    yūpaḥ kr̥ccʰraṃ bʰūtebʰyo 'bʰaya-dākṣiṇyām iti
	Sentence: d    kr̥tvā kratu mānasaṃ yāti kṣayaṃ budʰaḥ \30,8\.

     
    \end {itemize}
\end{frame}

\begin{frame}\frametitle{Some refernces}
\begin{itemize}
	\item summary based on the analysis by Bodewitz 1999 in Hindu Ahimsa and its root (Violence Denied: Violence, Non-Violence and the Rationalization of Violence 		in South Asian Cultural History, Brill) 
	\item Jan Gonda, 1959 in Four Studies in the Language of the Veda (The Hague: Mouton) raises the question as to "Why are áhiṁsā and similar concepts often 		  expressed in a negative form" (pages 95-117).
	
	\item Hans-Peter Schmidt  “The Origin of Ahiṃsā,” in Mélanges d’Indianisme à la mémoire de Louis Renou, Paris, 1968c, pp. 625-55.
	\item Hans-Peter Schmidt  “Ahimsā and Rebirth,” in M. Witzel, ed., Inside the Texts, Beyond the Texts: New Approaches to the Study of the Vedas, Harvard Oriental Series, Cambridge, MA, 1997, pp. 207-37.
	\item Schreiner, P. (1979). Gewaltlosigkeit und Tötungsverbot im Hinduismus.Angst und Gewalt, 287-308.
	\item Bürkle, H. 1. (1990). Ahiṃsā - Gewaltlosigkeit im Hinduismus und im Buddhismus.Mythos und Religion, 149-163.
	\item Thite, G. U. (1970). Animal-Sacrifice in the Brāhmaṇatexts. Numen, 17(2), 143–158. https://doi.org/10.2307/3269691
	\item Tull, H. W. (1996). The killing that is not killing: Men, cattle, and the origins of non-violence (ahisā) in the Vedic sacrifice.Indo-Iranian journal, 39(3), 223-244.
       \item A. Wezler: "Die wahren Speiserestesser ..." AWLM 1978; 
       \item A. Wezler: "The True Vigha sasin Remark on Mahabharata XII 214 and XII 11." Diamond Jub. Vol. Annals: BORI, 1978, pp.397-406; 
      \item A. Wezler: "Cattle, Field and Barley." Ady. Lib. Bul. 50. Madras 1986, pp. 431-477. 
      \item A. Welzer "On the Term antah samjha.." Annals: BORI 1987 (R.G. Bhandarkar Vol.), pp.111-131. 
	\item Wilhelm, F. (1991) Hunting and the concept of dharma. Panels of the VIIth World Sanskrit Conference (Kern Institute, Leiden : August 23-29. 1987 vol.
IX: Rules and Remedies in Classical Indian Law (ed . by Julia Leslie) :7 16. Leiden: Brill. 
	
\end{itemize}
\end{frame}

\begin{frame}\frametitle{Pacifism}
\begin{itemize}
\item what is emotionally \textbf {opposed to, or inconsistent} with these notions, a condition eagerly desired and difficult to obtain (Gonda)
\item Basham (1954: 123) "positive condemnations of war are rare in Indian literature"
	\item Arjuna in Mahabharata 12.15.20 \textit{na hi pasyami jivantam loke kamcid ahimsayti} stark contrast with his reluctance to fight in Gita \textbf{against 		relatives}.
	\end{itemize}
\end{frame}

\begin{frame}\frametitle{The Rejection of Hunting}
\begin{itemize}
\item The Brahmin and the Hunter (Book 3 Mahabharta)
\end{itemize}
\end{frame}

\begin{frame}\frametitle{Vegetarianism}
\begin{itemize}
\item  Mitra 1881:354-388 on "Beef in ancient India" - Meat eating was quiet acceptable
\end{itemize}
\end{frame}

\begin{frame}\frametitle{Animal Sacrifice}
\begin{itemize}
\item Krishna Lal 2005 वैदिक यज्ञों का स्वरूप Ram Lal Kapoor Trust
\item Yudhistra Mimamsaka Shabara Bhaṣya on Animal Sacrifice (to be searched)
\end{itemize}
\end{frame}

\begin{frame}\frametitle{}
\begin{itemize}
\item 
\end{itemize}
\end{frame}

\againframe{questions}

\end{document}
